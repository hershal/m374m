\documentclass[8pt]{article}

\usepackage{cheat-sheet-macros}

\begin{document}
\lhead{Hershal Bhave \quad\tiny{hb6279}}
\rhead{M374M Spring 2016 Exam 1 Cheat Sheet (Dr.\ Gonzalez)}

\begin{multicols}{2}
  \begin{description}
  \item[Pi Theorem] Given a physical equation $f(q_1,\ldots,q_n)=0$ where $q_i$
    are of $n$ physical variables, the above equation may be restated as
    $\tilde{f}(\pi_1,\ldots,\pi_m)$ where $\pi_i$ are dimensionless parameters
    constructed by $\pi_i=q_1^{\alpha_1}\cdots q_n^{\alpha_n}$
  \item[Proof of Pi Theorem] The general argument is exactly the same as the
    construction of dimensionless variables for $q$ physical quantities and $L$
    fundamental dimensions
    \begin{equation*}
        \begin{aligned}
          \pi &= q_1^{p_1}\cdots q_m^{p_m} \\
          [\pi] &= {(L_1^{a_{11}}\cdots L_n^{a_{n1}})}^{p_1}\cdots
          {(L_1^{a_{1m}}\cdots L_n^{a_{nm}})}^{p_1} \\
          1 &= L_1^{a_{11}p_1+\cdots+a_{1m}p_m}\cdots L_n^{a_{11}p_1+\cdots+a_{1m}p_m} \\
        \end{aligned}
    \end{equation*}
    Which turns into a homogenous system
    \begin{equation*}
      \begin{aligned}
        a_{11}p_1+\cdots+a_{1m}p_m &= 0 \\
        &\;\;\vdots \\
        a_{11}p_1+\cdots+a_{1m}p_m &= 0. \\
      \end{aligned}
    \end{equation*}
    This system may be represented by a matrix $A$, which has exactly
    $m-\text{rk}(A)$ independent solutions which form a null space for $A$.
    These independent solutions form appropriate dimesionless $\pi_i$.
  \item[Pi Function Manipulation] We (theoretically) may solve a system
    $f(\pi_1, \pi_2)$ for either $\pi_i$, resulting in a new relation
    $\pi_1=\tilde{f}(\pi_2)$. In the case when there is only one $\pi$, solving
    $f(\pi)=0$ for $\pi$ intuitively describes the roots of $f$, namely $\pi=C$
    where $C$ may be some number of constants.
  \item[Non-Dimensionalization] Given some IVP in the form $\od{y}{x}=f(x,y)$
    where $f(x,y)$ depends on some constants. To non-dimensionalize the problem,
    we choose dimensionless independent and dependent variables; select a
    characteristic $y$ and $x$ scale $y_c$ and $x_c$ formed from the constants
    in the problem (involves solving a nonhomogenous system). Then you may
    recast $\od{y}{x}$ using $\bar{y}=\frac{y}{y_c}$ and $\bar{x}=\frac{x}{x_c}$
    as $\od{\bar{y}}{\bar{x}}=\frac{x_c}{y_c}\od{y}{x}$ using the chain rule
    ($\od{\bar{y}}{\bar{x}}=\od{\bar{y}}{x}\frac{\dd{x}}{\dd{\bar{y}}}$). When a
    term is assumed to be ``small'', pick scales which do not involve constants
    within that term.
   \end{description}
 \end{multicols}
\end{document}
