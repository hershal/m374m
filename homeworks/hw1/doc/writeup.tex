\documentclass[12pt]{article}
\title{M374M Homework 1 \\
  \normalsize{\S~1.1 \#4$^1$, 5, 9$^2$, 13, 14$^3$}}
\author{Hershal Bhave (hb6279)}
\date{Due 2016-02-01}

\usepackage{macros}

\begin{document}
\maketitle

\section{\S~1.1}
\subsection{4$^1$}
\begin{Ex}
  In the blast wave problem, assume, instead of {\color{red}TODO}, a physical law
  in the form
  \begin{equation}
    g(t,r,\rho,e,P)=0,
  \end{equation}
  where $P$ is the ambient air pressure\footnote{$P=M L^{-1} T^{-2}$}. ``By
  inspection'', find two independent dimensionless parameters formed from $t$,
  $r$, $\rho$, $e$, and $P$. Naming the two dimensionless parameters $\pi_1$ and
  $\pi_2$ and assuming the law is equivalent to
  \begin{equation}
    f(\pi_1,\pi_2)=0,
  \end{equation}
  does it still follow that $r$ varies like the two-fifths power of $t$? Find
  the reduced form of $g(t,r,\rho,E,P)=0$ for $P\ne0$. In special case when
  $P=0$, deduce that $r=C(Et^2/\rho)^{1/5}$ as before.
  \begin{solution} \hfill \vspace{.75em} \\
    {\huge \color{red}TODO}
  \end{solution}
\end{Ex}

\subsection{5}
\begin{Ex}
  A physical system is described by a law $f(E,P,A)=0$, where $E$, $P$, and $A$
  are energy, pressure, and area, respectively. Show that $PA^{3/2}/E=\text{const}$.
  \begin{solution} \hfill \vspace{.75em} \\
    {\huge \color{red}TODO}
  \end{solution}
\end{Ex}

\subsection{9$^2$}
\begin{Ex}
  In modeling the digestion process in the insects, it is believed that
  digestion yield rate $Y$, in mass per time, is related to the concentration
  $C$, of the limiting nutrient, the residence time $T$ in the gut, the gut
  volume $V$, and the rate of nutrient breakdown $r$, given the mass per time
  per volume. Show that for fixed $T$, $r$, $C$, the yield is positively related
  to the gut volume. For fixed $(T,r,C)$, show that $Y$ must be proportional to
  $V$. The dimensions of concentration are $M/L^3$.
  \begin{solution} \hfill \vspace{.75em} \\
    {\huge \color{red}TODO}
  \end{solution}
\end{Ex}

\subsection{13}
\begin{Ex}
  Imagine an experiment where we set up a line of dominos with spacing $d$
  between them. Further, we assume a typical domino has height $h$ and thickness
  $\tau$. We seek a formula that relates these quantities, the gravitational
  constant $g$, and the velocity $v$.

  \begin{enumerate}
  \item Use dimensional analysis to show
    that $$v=\sqrt{gh}F\left(\frac{d}{h},\frac{\tau}{h}\right).$$
    Assume $\tau/h$ is very small and can be neglected. What does the law become?
  \item Experiments have been performed that show the graph of $v/\sqrt{gh}$ vs.
    $d/h$ is approximately constant, $1.5$, for $d/h$ varying over the range $0$
    to $0.8$. Using $h=0.05$ meters, what is the velocity of the toppling dominos?
  \end{enumerate}
  \begin{solution} \hfill \vspace{.75em} \\
    {\huge \color{red}TODO}
  \end{solution}
\end{Ex}

\subsection{14$^3$}
\begin{Ex}
  A perfect gas in equilibrium has specific energy $E$ (energy per mass),
  temperature $T$, and Boltzmann constant $k$ (specific energy per degress).
  Derive a functional relationship of the form $E=f(k,T)$. Assuming a law
  $F(E,T,k)=0$, derive an equivalent reduce law and show that $E=ckT$ for some
  constant $c$.
  \begin{solution} \hfill \vspace{.75em} \\
    {\huge \color{red}TODO}
  \end{solution}
\end{Ex}

\section{Programming Minilab}
A model for an ideal pendulum released from rest is

\begin{equation}
  \label{eq:minilab-pendulum-model}
   \left\{
  \begin{aligned}
    l\ddot{\theta}+g\sin\theta = 0, \quad & t\ge0 \\
    \dot{\theta} = 0, \quad & t = 0 \\
    \theta = \theta_0, \quad & t = 0.
  \end{aligned}\right.
\end{equation}

Here $\theta$ is the pendulum angle, $l$ is the pendulum length, $g$ is the
gravitational acceleration constant, $t$ is time, and over-dots denote time
derivatives. A dimensional analysis of \cref{eq:minilab-pendulum-model} reveals
that the general solution can be written in the form

\begin{equation}
  \label{eq:minilab-general-solution}
  \theta = \tilde{f}(t\sqrt{g/l}, \;\theta_0)
\end{equation}

for some function $\tilde{f}$. Here we investigate various consequences of
\cref{eq:minilab-general-solution}. Below we use meters and seconds as our units
for length and time.

\subsection{}
\label{sec:minilab-part-1}
\begin{Ex}
  For the initial conditions $\theta_0=3\pi/8$ and interval $t\in[0,2.5]$,
  superimpose plots of $\theta$ versus $t$ for different values of $g$ and $l$,
  say, $(g,l)=(10, 0.25), (10, 0.5), (10, 1)$. Based on the general solution
  $\theta = f(t,g,l,\theta_0)$, briefly explain why different values of $g$ and
  $l$ produce different curves.
  \begin{solution} \hfill \vspace{.75em} \\
    {\huge \color{red}TODO}
  \end{solution}
\end{Ex}

\subsection{}
Repeat \cref{sec:minilab-part-1} but now plot $\theta$ vs $t\sqrt{g/l}$. Based
on the from solution in , briefly
explain why different values of g and l produce the same curve (or portion thereof). If θ 0 were
changed in addition to g and l, would we still get this same curve?

\end{document}
