\documentclass[12pt]{article}
\title{M374M Homework 1 \\
  \normalsize{\S~1.1 \#4$^1$, 5, 9$^2$, 13, 14$^3$}}
\author{Hershal Bhave (hb6279)}
\date{Due 2016-02-01}

\usepackage{macros}

\begin{document}
\maketitle

\section{\S~1.1}
\subsection{4$^1$}
  \subsubsection*{Problem}
  In the blast wave problem, assume, instead of {\color{red}TODO}, a physical law
  in the form
  \begin{equation}
    g(t,r,\rho,E,P)=0,
  \end{equation}
  where $P$ is the ambient air pressure\footnote{$P=M L^{-1} T^{-2}$}. ``By
  inspection'', find two independent dimensionless parameters formed from $t$,
  $r$, $\rho$, $E$, and $P$. Naming the two dimensionless parameters $\pi_1$ and
  $\pi_2$ and assuming the law is equivalent to
  \begin{equation}
    f(\pi_1,\pi_2)=0,
  \end{equation}
  does it still follow that $r$ varies like the two-fifths power of $t$?

  \subsubsection*{Remarks}
  Find the reduced form of $g(t,r,\rho,E,P)=0$ for $P\ne0$. In special case when
  $P=0$, deduce that $r=C(Et^2/\rho)^{1/5}$ as before.

  \subsubsection*{Solution}
  In the case where $P\ne0$, we will find exponents where $g(t,r,\rho,E,P)=0$.
  Utilizing the mapping in \cref{tab:4-var-mappings}, we obtain the relation in
  \cref{eq:4-p-ne-step-1}.

  \begin{figure}
    \centering
    \label{tab:4-var-mappings}
    \begin{tabularx}{0.5\textwidth}{XXX}
      Variable & Unit & Exponent \\ \hline
      $t$ & $T$ & $a$ \\
      $r$ & $L$ & $b$ \\
      $\rho$ & $ML^{-3}$ & $c$ \\
      $E$ & $ML^{2}L^{-2}$ & $d$ \\
      $P$ & $ML^{-1}T^{-2}$ & $e$ \\
    \end{tabularx}
    \caption{Variable mappings for \#4}
  \end{figure}

  \begin{equation}
    \label{eq:4-p-ne-step-1}
    \begin{aligned}
      0 &= (T)^a(L)^b(ML^{-3})^c(ML^2T^{-2})^d(ML^{-1}T^{-2})^e\\
        &= M^{c+d+e}L^{b-3c+2d-e}T^{a-2d-2e} \\
    \end{aligned}
  \end{equation}

  We can now isolate the exponents for each unit.

  \begin{equation}
    \label{eq:4-p-ne-step-2}
    \begin{aligned}
      c + d + e &= 0 \\
      b - 3c + 2d - e &= 0 \\
      a - 2d - 2e &= 0 \\
    \end{aligned}
  \end{equation}

  \Cref{eq:4-p-ne-step-2} includes five unknowns in three equations. Thus we
  must choose values for at least two variables. We will pick values for two variables

  \begin{equation}
    \implies d=1, \quad e=1.
  \end{equation}

  \begin{equation}
    \begin{aligned}
      c+d+e &= 0\\
      c + 2 &= 0\\
      \implies c &= -2.\\
    \end{aligned}
  \end{equation}

  Furthermore,

  \begin{equation}
    \begin{aligned}
      b-3c+2d-e&=0\\
      b-3(-2)+2(1)-(1)&=0 \\
      b-6+2-1&=0 \\
      \implies b&=5.\\
    \end{aligned}
  \end{equation}

  Which implies

  \begin{equation}
    \begin{aligned}
      a-2d-2e&=0\\
      a-2(1)-2(1)&=0\\
      \implies a&=4.\\
    \end{aligned}
  \end{equation}

  From this information, we can construct the final answer for $P\ne0$, shown in
  \cref{eq:4-p-ne-sol}.

  \begin{equation}
    \label{eq:4-p-ne-sol}
      \boxed{
        t^4r^5\rho^{-2}EP=C
      }
    \end{equation}
  {\huge \color{red}TODO}

\subsection{5}
  \subsubsection*{Problem}
  A physical system is described by a law $f(E,P,A)=0$, where $E$, $P$, and $A$
  are energy, pressure, and area, respectively. Show that $PA^{3/2}/E=\text{const}$.

  \subsubsection*{Solution}
  {\huge \color{red}TODO}

\subsection{9$^2$}
  \subsubsection*{Problem}
  In modeling the digestion process in the insects, it is believed that
  digestion yield rate $Y$, in mass per time, is related to the concentration
  $C$, of the limiting nutrient, the residence time $T$ in the gut, the gut
  volume $V$, and the rate of nutrient breakdown $r$, given the mass per time
  per volume. Show that for fixed $T$, $r$, $C$, the yield is positively related
  to the gut volume.

  \subsubsection*{Remarks}
  For fixed $(T,r,C)$, show that $Y$ must be proportional to
  $V$. The dimensions of concentration are $M/L^3$.

  \subsubsection*{Solution}
  {\huge \color{red}TODO}

\subsection{13}
  \subsubsection*{Problem}
  Imagine an experiment where we set up a line of dominos with spacing $d$
  between them. Further, we assume a typical domino has height $h$ and thickness
  $\tau$. We seek a formula that relates these quantities, the gravitational
  constant $g$, and the velocity $v$.

  \begin{enumerate}
  \item Use dimensional analysis to show
    that $$v=\sqrt{gh}F\left(\frac{d}{h},\frac{\tau}{h}\right).$$
    Assume $\tau/h$ is very small and can be neglected. What does the law become?
  \item Experiments have been performed that show the graph of $v/\sqrt{gh}$ vs.
    $d/h$ is approximately constant, $1.5$, for $d/h$ varying over the range $0$
    to $0.8$. Using $h=0.05$ meters, what is the velocity of the toppling dominos?
  \end{enumerate}

  \subsubsection*{Solution}
  {\huge \color{red}TODO}

\subsection{14$^3$}
  \subsubsection*{Problem}
  A perfect gas in equilibrium has specific energy $E$ (energy per mass),
  temperature $T$, and Boltzmann constant $k$ (specific energy per degress).
  Derive a functional relationship of the form $E=f(k,T)$.

  \subsubsection*{Remarks}
  Assuming a law $F(E,T,k)=0$, derive an equivalent reduce law and show that
  $E=ckT$ for some constant $c$.

  \subsubsection*{Solution}
  {\huge \color{red}TODO}

\section{Programming Minilab}
A model for an ideal pendulum released from rest is

\begin{equation}
  \label{eq:minilab-pendulum-model}
   \left\{
  \begin{aligned}
    &\ell\ddot{\theta}+g\sin\theta = 0, \quad & t\ge0 \\
    &\dot{\theta} = 0, \quad & t = 0 \\
    &\theta = \theta_0, \quad & t = 0.
  \end{aligned}\right.
\end{equation}

Here $\theta$ is the pendulum angle, $\ell$ is the pendulum length, $g$ is the
gravitational acceleration constant, $t$ is time, and over-dots denote time
derivatives. A dimensional analysis of \cref{eq:minilab-pendulum-model} reveals
that the general solution can be written in the form

\begin{equation}
  \label{eq:minilab-general-solution}
  \theta = \tilde{f}(t\sqrt{g/\ell}, \;\theta_0)
\end{equation}
for some function $\tilde{f}$. Here we investigate various consequences of
\cref{eq:minilab-general-solution}. Below we use meters and seconds as our units
for length and time.

\subsection{}
  \label{sec:minilab-part-1}
  \subsubsection*{Problem}
  For the initial conditions $\theta_0=3\pi/8$ and interval $t\in[0,2.5]$,
  superimpose plots of $\theta$ versus $t$ for different values of $g$ and $\ell$,
  say, $(g,\ell)=(10, 0.25), (10, 0.5), (10, 1)$. Based on the general solution
  $\theta = f(t,g,\ell,\theta_0)$, briefly explain why different values of $g$ and
  $\ell$ produce different curves.

  \subsubsection*{Solution}
  {\huge \color{red}TODO}

\subsection{}
  \label{sec:minilab-part-2}
  \subsubsection*{Problem}
  Repeat \cref{sec:minilab-part-1} but now plot $\theta$ vs $t\sqrt{g/\ell}$.
  Based on the from solution in \cref{eq:minilab-general-solution}, briefly
  explain why different values of $g$ and $\ell$ produce the same curve (or
  portion thereof). If $\theta_0$ were changed in addition to $g$ and $\ell$,
  would we still get this same curve?

  \subsubsection*{Solution}
    {\huge \color{red}TODO}

\subsection{}
  \subsubsection*{Problem}
  Let $T$ be the period of the pendulum motion for given $\theta_0$, $g$, $\ell$.
  Using the form of the solution in \cref{eq:minilab-general-solution}, show
  $T=\sqrt{\ell/g}h(\theta_0)$ for some function $h$. Does this agree with the
  observations in \cref{sec:minilab-part-2}? Specifically, for fixed $\theta_0$
  and $g$, does $T$ increase by a factor of two when $\ell$ is increased by a
  factor of four?

  \subsubsection*{Solution}
  {\huge \color{red}TODO}
\end{document}
