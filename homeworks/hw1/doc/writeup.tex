\documentclass[12pt]{article}
\title{M374M Homework 1 \\
  \normalsize{\S~1.1 \#4$^1$, 5, 9$^2$, 13, 14$^3$}}
\author{Hershal Bhave (hb6279)}
\date{Due 2016-02-01}

\usepackage{multicol}
\usepackage{float}
\usepackage[in]{fullpage}
\usepackage{xcolor}
\usepackage{mdframed}
\usepackage{tabularx}
\usepackage{rotating}
\usepackage{mathtools}
\usepackage{amssymb}
\usepackage{cleveref}
\usepackage{graphics}
\usepackage{caption}
\usepackage{wrapfig}
\usepackage{subcaption}
\usepackage[nosolutionfiles]{answers}
\usepackage[acronym]{glossaries}
\usepackage{listings}

\newenvironment{Ex}{\textbf{Problem}\vspace{.75em}\\}{}
\Newassociation{solution}{Soln}{Answers}
\pagebreak[3]
\newcommand{\Opentesthook}[2]{\Writetofile{#1}{\protect\section{#1: #2}}}
\renewcommand{\Solnlabel}[1]{\textbf{Solution}\quad}
\newcommand{\dd}[1]{\mathrm{d}{#1}}
\newcommand{\ddt}[1]{\frac{\dd{}}{\dd{#1}}}
\newcommand{\dddt}[1]{\frac{\dd{}^2}{\dd{#1}^2}}
\newcommand{\aoneonematrix}{
  \begin{bmatrix}
    1 & 0.25 & 0.75 \\
    1 & 0    & 0.5 \\
    1 & 0    & 1 \\
  \end{bmatrix}
}
\newcommand{\aonetwomatrix}{
  \begin{bmatrix}
    1 & 0.25 & 0.75 \\
    1 & 0    & 0.5  \\
    1 & 0.5  & 0.5  \\
  \end{bmatrix}
}
\newcommand{\bonematrix}[2]{
  \begin{bmatrix}
    a_{#1}^{(#2)} \\
    b_{#1}^{(#2)} \\
    c_{#1}^{(#2)} \\
  \end{bmatrix}
}
\newcommand{\setupaonematrices}[2]{
\aoneonematrix \bonematrix{#1}{#2}
}
\newcommand{\setupatwomatrices}[2]{
\aonetwomatrix \bonematrix{#1}{#2}
}

\definecolor{mygreen}{rgb}{0,0.6,0}
\definecolor{mygray}{rgb}{0.5,0.5,0.5}
\definecolor{mymauve}{rgb}{0.58,0,0.82}

\lstloadlanguages{Octave,C++,Perl}

\lstset{
  backgroundcolor=\color{white},
  basicstyle=\scriptsize\ttfamily,
  breakatwhitespace=false,
  breaklines=true,
  captionpos=b,
  commentstyle=\color{mygreen},
  deletekeywords={...},
  escapeinside={\%*}{*)},
  extendedchars=true,
  frame=single,
  keywordstyle=\color{blue},
  % language=Octave,
  % numbers=left,
  % numbersep=5pt,
  % numberstyle=\tiny\color{mygray},
  rulecolor=\color{black},
  showspaces=false,
  showstringspaces=false,
  showtabs=false,
  % stepnumber=2,
  stringstyle=\color{mymauve},
  tabsize=2,
  title=\lstname,
  columns=fullflexible,
}

\relpenalty=10000
\binoppenalty=10000

\begin{document}
\maketitle

\section{\S~1.1}
\subsection{4$^1$}
\begin{Ex}
  In the blast wave problem, assume, instead of {\color{red}TODO}, a physical law
  in the form
  \begin{equation}
    g(t,r,\rho,e,P)=0,
  \end{equation}
  where $P$ is the ambient air pressure\footnote{$P=M L^{-1} T^{-2}$}. ``By
  inspection'', find two independent dimensionless parameters formed from $t$,
  $r$, $\rho$, $e$, and $P$. Naming the two dimensionless parameters $\pi_1$ and
  $\pi_2$ and assuming the law is equivalent to
  \begin{equation}
    f(\pi_1,\pi_2)=0,
  \end{equation}
  does it still follow that $r$ varies like the two-fifths power of $t$? Find
  the reduced form of $g(t,r,\rho,E,P)=0$ for $P\ne0$. In special case when
  $P=0$, deduce that $r=C(Et^2/\rho)^{1/5}$ as before.
  \begin{solution} \hfill \vspace{.75em} \\
    {\huge \color{red}TODO}
  \end{solution}
\end{Ex}

\subsection{5}
\begin{Ex}
  A physical system is described by a law $f(E,P,A)=0$, where $E$, $P$, and $A$
  are energy, pressure, and area, respectively. Show that $PA^{3/2}/E=\text{const}$.
  \begin{solution} \hfill \vspace{.75em} \\
    {\huge \color{red}TODO}
  \end{solution}
\end{Ex}

\subsection{9$^2$}
\begin{Ex}
  In modeling the digestion process in the insects, it is believed that
  digestion yield rate $Y$, in mass per time, is related to the concentration
  $C$, of the limiting nutrient, the residence time $T$ in the gut, the gut
  volume $V$, and the rate of nutrient breakdown $r$, given the mass per time
  per volume. Show that for fixed $T$, $r$, $C$, the yield is positively related
  to the gut volume. For fixed $(T,r,C)$, show that $Y$ must be proportional to
  $V$. The dimensions of concentration are $M/L^3$.
  \begin{solution} \hfill \vspace{.75em} \\
    {\huge \color{red}TODO}
  \end{solution}
\end{Ex}

\subsection{13}
\begin{Ex}
  Imagine an experiment where we set up a line of dominos with spacing $d$
  between them. Further, we assume a typical domino has height $h$ and thickness
  $\tau$. We seek a formula that relates these quantities, the gravitational
  constant $g$, and the velocity $v$.

  \begin{enumerate}
  \item Use dimensional analysis to show
    that $$v=\sqrt{gh}F\left(\frac{d}{h},\frac{\tau}{h}\right).$$
    Assume $\tau/h$ is very small and can be neglected. What does the law become?
  \item Experiments have been performed that show the graph of $v/\sqrt{gh}$ vs.
    $d/h$ is approximately constant, $1.5$, for $d/h$ varying over the range $0$
    to $0.8$. Using $h=0.05$ meters, what is the velocity of the toppling dominos?
  \end{enumerate}
  \begin{solution} \hfill \vspace{.75em} \\
    {\huge \color{red}TODO}
  \end{solution}
\end{Ex}

\subsection{14$^3$}
\begin{Ex}
  A perfect gas in equilibrium has specific energy $E$ (energy per mass),
  temperature $T$, and Boltzmann constant $k$ (specific energy per degress).
  Derive a functional relationship of the form $E=f(k,T)$. Assuming a law
  $F(E,T,k)=0$, derive an equivalent reduce law and show that $E=ckT$ for some
  constant $c$.
  \begin{solution} \hfill \vspace{.75em} \\
    {\huge \color{red}TODO}
  \end{solution}
\end{Ex}

\end{document}
